\documentclass[paper=a4, fontsize=11pt]{scrartcl} % A4 paper and 11pt font size

\usepackage[T1]{fontenc} % Use 8-bit encoding that has 256 glyphs
\usepackage[italian]{babel} % English language/hyphenation
\usepackage{amsmath,amsfonts,amsthm} % Math packages


\usepackage{sectsty} % Allows customizing section commands
\allsectionsfont{\centering \normalfont\scshape} % Make all sections centered, the default font and small caps

\usepackage{fancyhdr} % Custom headers and footers
\pagestyle{fancyplain} % Makes all pages in the document conform to the custom headers and footers
\fancyhead{} % No page header - if you want one, create it in the same way as the footers below
\fancyfoot[L]{} % Empty left footer
\fancyfoot[C]{} % Empty center footer
\fancyfoot[R]{\thepage} % Page numbering for right footer
\renewcommand{\headrulewidth}{0pt} % Remove header underlines
\renewcommand{\footrulewidth}{0pt} % Remove footer underlines
\setlength{\headheight}{13.6pt} % Customize the height of the header

\numberwithin{equation}{section} % Number equations within sections (i.e. 1.1, 1.2, 2.1, 2.2 instead of 1, 2, 3, 4)
\numberwithin{figure}{section} % Number figures within sections (i.e. 1.1, 1.2, 2.1, 2.2 instead of 1, 2, 3, 4)
\numberwithin{table}{section} % Number tables within sections (i.e. 1.1, 1.2, 2.1, 2.2 instead of 1, 2, 3, 4)

\setlength\parindent{0pt} % Removes all indentation from paragraphs - comment this line for an assignment with lots of text

%----------------------------------------------------------------------------------------
%	TITLE SECTION
%----------------------------------------------------------------------------------------

\newcommand{\horrule}[1]{\rule{\linewidth}{#1}} % Create horizontal rule command with 1 argument of height

\title{
\normalfont \normalsize
\textsc{documento requisiti} \\ [25pt] % Your university, school and/or department name(s)
\horrule{0.5pt} \\[0.4cm] % Thin top horizontal rule
\huge Gestionale Aziendale SpeedCasa \\ % The assignment title
\horrule{2pt} \\[0.5cm] % Thick bottom horizontal rule
}

\author{Dr. Nicola Pancheri}

\date{\normalsize\today} % Today's date or a custom date

\begin{document}

\maketitle % Print the title

\section{Requisiti}

%------------------------------------------------

Si vuole realizzare un sistema informativo per gestire le informazioni relative all’attività svolta da
l’azienda di ristrutturazioni \textit{SpeedCasa}.
I dipendenti di tal azienda faranno accesso a tal sistema tramite
un'applicazione realizzata appositamente con lo scopo di coordinare gli uffici.
L'applicazione sar\'a una web-app accessibile tramite browser; i dipendenti
dovranno essere in grado di accedervi anche da casa.
Vi sono tre tipi di ufficio: ufficio commerciale, ufficio tecnico e ufficio capi cantiere.
I dipendenti, per accedere alle funzionalit\'a dell'applicazione e ai dati contenuti
nel database, dovranno prima essere registrati.
La registrazione di un dipendete viene effettuata da un apposito responsabile.
I dipendenti vengono registrati memorizzando: nome, cognome, un'identificativo di login con
la relativa password, l'ufficio, una foto personale (opzionale).
Ogni dipendente dovr\'a effettuare un login prima di poter aver accesso al sistema.
Ogni dipendente pu\'o avere accesso solo ad un gruppo ristretto di informazioni e funzionalit\'a
in base al ufficio a cui appartiene;
Tramite l'applicazione i dipendenti, a prescindere dall'ufficio d'appertenenza, devono essere
in grado di comunicare tra loro.
Successivamente al fase di login al dipendente viene presentanta la sua pagina di lavoro
personalizzata in base all'ufficio di appartenenza.
A prescindere dall'ufficio deve essere disponibile una finestra \textit{cose da fare}:
tramite questa finestra un dipendente deve essere in grado di visualizzare
i suoi impegni giornalieri. Tali impegni devono essere impostati precedente
dall'dipendete stesso.


\subsection{Base di Dati}

\begin{align}

\end{align}

%------------------------------------------------

\subsection{Applicazione}

\lipsum[3] % Dummy text

\paragraph{Heading on level 4 (paragraph)}

\lipsum[6] % Dummy text

%----------------------------------------------------------------------------------------
%	PROBLEM 2
%----------------------------------------------------------------------------------------

\section{Lists}

%------------------------------------------------

\subsection{Example of list (3*itemize)}
\begin{itemize}
	\item First item in a list
		\begin{itemize}
		\item First item in a list
			\begin{itemize}
			\item First item in a list
			\item Second item in a list
			\end{itemize}
		\item Second item in a list
		\end{itemize}
	\item Second item in a list
\end{itemize}

%------------------------------------------------

\subsection{Example of list (enumerate)}
\begin{enumerate}
\item First item in a list
\item Second item in a list
\item Third item in a list
\end{enumerate}

%----------------------------------------------------------------------------------------

\end{document}
